\section{Examples of Cache Coherence}
So now we will try to simulate how what we have seen in the previous section is working concretely. For this we will use the \important{MESI Protocol}

\begin{parag}{MESI Protocol}
    Consider a \important{system of four processors} P0 to P3, each having its own cache C0 to C3 respectively. The \important{MESI protocol} is used to maintain coherence among the different caches.\\

	With this protocol, a cache line can be in one of four states:
	\begin{itemize}
	    \item Modified
	    \item Exclusive
	    \item Shared
	    \item Invalid
	\end{itemize}
	Let us take a look at the finite state machine of the following protocol (the dotted lines are the snooping lines)
\end{parag}


	\begin{center}
			
\begin{tikzpicture}[
	state/.style=
	{circle, minimum size=0cm, draw, align=center, ->, >=stealth', auto, semithick },
	>=Stealth
]

% Nodes
\node[state]  at (0, 0) (invalid)   {Invalid};
\node[state]  at (0, 4) (shared)    {Shared};
\node[state]  at (0, 8) (exclusive) {Exclusive};
\node[state]  at (0, 12) (modified)  {Modified};


\draw[->] (invalid.west) edge[bend left] node[left] {PrWr / BusRdX} (modified.west);
\draw[->] (invalid) edge[bend left] node[left] {PrRd / BusRd, BusShr='1'} (shared);
\draw[->] (invalid) edge[bend left] node[left] {PrRd / BusRd, BusShr='0'} (exclusive);

\draw[->] (shared) edge[bend left] node[left] {PrWr / BusRdX} (modified);
\draw[->] (shared) edge[loop below] node[below] {PrRd / --} (shared);

\draw[->] (exclusive) edge[bend left] node[right] {PrWr / --} (modified);
\draw[->] (exclusive) edge[loop below] node {PrRd / --} (exclusive);

\draw[->] (modified) edge[loop above] node {PrRd / --, PrWr / --} (modified);

% -------- Bus transitions (dotted, snooping

\draw[dotted, thick, ->] (modified) edge[bend left] node[right] {BusRd / BusWr + BusShr} (shared);
\draw[dotted, thick,->] (modified.east) edge[bend left] node[right] {BusRdX / BusWr} (invalid.east);

\draw[dotted, thick,->] (exclusive) edge[bend left] node[right] {BusRd / BusShr} (shared);
\draw[dotted, thick,->] (exclusive) edge[bend right=25] node[left] {BusRdX / --} (invalid);

\draw[dotted, thick,->] (shared) edge[bend left] node[right] {BusRdX / --} (invalid);
\draw[dotted, thick,->] (invalid) edge[loop below] node[below] {BusRd / --, BusRdX / --} (invalid);

\end{tikzpicture}
\begin{parag}{$ $}
    

\begin{subparag}{Remark}
    I tried to make it readable but:
	\begin{itemize}
		\item Each node (little text for each arrow) is at the midle of the arrow.
		\item If the arrow is on the left \textrightarrow the node is on the left of the arrows
		\item If the arrow is on the right \textrightarrow the node is on the right of the arrow
		\item If the arrow is a loop then it will by below if the loop is below or above otherwise
		\item The dotted lines represent the bus action (the normal lines represent the processor action).
	\end{itemize}
	
\end{subparag}
\end{parag}


	\end{center}

	The difference is now that we can be exclusive dirty (the modified state) \important{or} exclusive clean.
	So intuitively we have:
	\begin{itemize}
		\item \texttt{Invalid} \textrightarrow "I do not have a valid copy of X"
			\begin{itemize}
				\item The cache line is empty or outdated
				\item Any read/write needs to fetch data from elsewhere
			\end{itemize}
		\item \texttt{Shared} \textrightarrow "I have a clean copy of X, and \important{other caches may also have it}"
			\begin{itemize}
				\item Same value exists in memory
				\item Multiple caches may hold X
				\item \important{Read-only} state
				\item Writing is \important{not allowed} without invalidating others
			\end{itemize}
		\item \texttt{Exclusive} \textrightarrow "I have the \important{only} copy of X, and it matches memory"
			\begin{itemize}
				\item No other cache has X
				\item Memory is up to date
				\item Can silently upgrade to Modified on write
			\end{itemize}
		\item Modified \textrightarrow "I have the only copy of X, and memory is outdated"
			\begin{itemize}
				\item This cache has changed the value
				\item Memory must be updated before other can read
				\item \important{Dirty} data
			\end{itemize}
	\end{itemize}
	Without the \important{Exclusive} state a read would always go:
	\begin{center}
	    Invalid \textrightarrow Shared \textrightarrow Modified
	\end{center}
	The MESI optimizes it to:
	\begin{center}
	    Invalid \textrightarrow Exclusive \textrightarrow Modified
	\end{center}
	
	
	
	\begin{parag}{MESI Protocol Shared and Exclusive States}
		In the MESI protocol, the following needs to be considered when a processor reads a data item that was previously not in his own cache:
		\begin{itemize}
			\item If the data is already stored in another processor cache, then the status of the cache line holding this data is set to the \important{Shared} state
			\item If the data is not already stored in any other cache, then the status of the cache line which will hold this data is set to the \important{Exclusive} state.
		\end{itemize}
	\end{parag}
	\begin{parag}{MESI Protocol BusRd and BusShr}
	    To know if the data is already in another cache or not, when one processor reads a data item, with a BusRd transaction, the other caches must indicate if they hold a copy of the data item by setting a special bus signal called BusShr to '1' \\

		If the signal is '0', we know that the particular data item is not held in any other cache
	\end{parag}
	
	
	\begin{parag}{Part I}
		\begin{itemize}
			\item In the following diagrams, we are given the current state (labeled as \important{Old}) of a specific cache line in the four caches C0 to C3
			\item For each cache, the address of the data saved in that cache line, as well as its state, is specified
			\item When one of the processors (P0 to P3) executes an instruction that requires a memory access, then a state change is triggered in one or more of the caches (C0 to C3)
			\item This change is shown, for one cache, under the label \important{New}
		\end{itemize}
		We are now required to fill the rest of the template as follows:
		\begin{itemize}
			\item Specify the new state/address of the remaining caches
			\item Specify the memory acess that caused these changes; our answer should be in the form
				\begin{center}
				    P<n>:Load/Store<address>
				\end{center}
				Where P<n> is the processor that executed the memory operation, Load/Store is the type of memory access and <address> is the address of the data to be loaded/stored
		\end{itemize}
		Note that there might be up to 3 possible answers for each case and that \important{you are required to list all possible answers} in the provided space
		
	\end{parag}
	
	\begin{center}
\begin{tabular}{|c|cc|cc|cc|cc|c|}
\hline
 & \multicolumn{8}{c|}{Cache} &  \\
\cline{2-9}
 & \multicolumn{2}{c|}{Cache C0}
 & \multicolumn{2}{c|}{Cache C1}
 & \multicolumn{2}{c|}{Cache C2}
 & \multicolumn{2}{c|}{Cache C3}
 & P\textless n\textgreater : L/S \textless addr\textgreater \\
\cline{2-9}
 & State & Address
 & State & Address
 & State & Address
 & State & Address
 & \\
\hline
Old
 & I & 0x1000
 & S & 0x1000
 & S & 0x2000
 & S & 0x1000
 & \\
\hline
New
 & S & 0x1000
 &   &
 &   &
 &   &
 & \\
\cline{2-9}
 & S & 0x1000
 &   &
 &   &
 &   &
 & \\
\cline{2-9}
 & S & 0x1000
 &   &
 &   &
 &   &
 & \\
\hline
\end{tabular}
\end{center}
The only way here to go from Invalid to shared for the C0 cache is by doing a processor read \textrightarrow P0:Load 0x100. This way all the other cache go from shared to shared still.


	\begin{center}
\begin{tabular}{|c|cc|cc|cc|cc|c|}
\hline
 & \multicolumn{8}{c|}{Cache} &  \\
\cline{2-9}
 & \multicolumn{2}{c|}{Cache C0}
 & \multicolumn{2}{c|}{Cache C1}
 & \multicolumn{2}{c|}{Cache C2}
 & \multicolumn{2}{c|}{Cache C3}
 & P\textless n\textgreater : L/S \textless addr\textgreater \\
\cline{2-9}
 & State & Address
 & State & Address
 & State & Address
 & State & Address
 & \\
\hline
Old
 & I & 0x1000
 & S & 0x1000
 & S & 0x2000
 & S & 0x1000
 & \\
\hline
New
 & S & 0x1000
 & S  & 0x1000
 &  S & 0x2000
 &  S & 0x1000
 & P0:Load 0x1000 \\
\cline{2-9}
 & S & 0x1000
 &   &
 &   &
 &   &
 & \\
\cline{2-9}
 & S & 0x1000
 &   &
 &   &
 &   &
 & \\
\hline
\end{tabular}
\end{center}



So now here is a couple of examples:




	\begin{center}
\begin{tabular}{|c|cc|cc|cc|cc|c|}
\hline
 & \multicolumn{8}{c|}{Cache} &  \\
\cline{2-9}
 & \multicolumn{2}{c|}{Cache C0}
 & \multicolumn{2}{c|}{Cache C1}
 & \multicolumn{2}{c|}{Cache C2}
 & \multicolumn{2}{c|}{Cache C3}
 & P\textless n\textgreater : L/S \textless addr\textgreater \\
\cline{2-9}
 & State & Address
 & State & Address
 & State & Address
 & State & Address
 & \\
\hline
Old
 & E & 0x1000
 & I & 0x1000
 & S & 0x2000
 & I & 0x1000
 & \\
\hline
New
 & S & 0x1000
 &   & 
 &  & 
 &  &
 & \\
\cline{2-9}
 & S & 0x1000
 &   &
 &   &
 &   &
 & \\
\cline{2-9}
 & S & 0x1000
 &   &
 &   &
 &   &
 & \\
\hline
\end{tabular}
\end{center}
So here we go from exclusive to shared. The only way of doing that is a BusRd. If we see it without the finite state machine, remember that exclusive means "I am the only one having the data", shared means "I have the correct data and at least another ones has it". For us to confidently go into shared we need the information that another processor has the data \textrightarrow a Bus read.


Now we go in our finite state machine diagram and we look for all the possible way of lauching a BusRd. For us, we will chosse P1:Load 0x1000. The only way for the C0 to go to the shared state, is to have another processor launch a BusRd, The way of doing that is for instance having the P1 to read the value at 0x1000:
	\begin{center}
\begin{tabular}{|c|cc|cc|cc|cc|c|}
\hline
 & \multicolumn{8}{c|}{Cache} &  \\
\cline{2-9}
 & \multicolumn{2}{c|}{Cache C0}
 & \multicolumn{2}{c|}{Cache C1}
 & \multicolumn{2}{c|}{Cache C2}
 & \multicolumn{2}{c|}{Cache C3}
 & P\textless n\textgreater : L/S \textless addr\textgreater \\
\cline{2-9}
 & State & Address
 & State & Address
 & State & Address
 & State & Address
 & \\
\hline
Old
 & E & 0x1000
 & I & 0x1000
 & S & 0x2000
 & I & 0x1000
 & \\
\hline
New
 & S & 0x1000
 & S  &  0x1000
 & S & 0x2000
 &  I & --
 & P1:Load 0x1000 \\
\cline{2-9}
 & S & 0x1000
 &   &
 &   &
 &   &
 & \\
\cline{2-9}
 & S & 0x1000
 &   &
 &   &
 &   &
 & \\
\hline
\end{tabular}
\end{center}
As we can do the same thing from the third processor (both are totaly the same). And as everyone will be in the shared state, someone else would also need to read from memory P2.\\
As the C3 processor has the same state as the C1 processor, we also have the \texttt{P3:Load 0x1000}. 

If we take a look at the C2 cache it is shared \important{but for the address 0x2000}. this totally equivalent as being invalid for 0x1000 right? So might as well do the same thing as the other:
	\begin{center}
\begin{tabular}{|c|cc|cc|cc|cc|c|}
\hline
 & \multicolumn{8}{c|}{Cache} &  \\
\cline{2-9}
 & \multicolumn{2}{c|}{Cache C0}
 & \multicolumn{2}{c|}{Cache C1}
 & \multicolumn{2}{c|}{Cache C2}
 & \multicolumn{2}{c|}{Cache C3}
 & P\textless n\textgreater : L/S \textless addr\textgreater \\
\cline{2-9}
 & State & Address
 & State & Address
 & State & Address
 & State & Address
 & \\
\hline
Old
 & E & 0x1000
 & I & 0x1000
 & S & 0x2000
 & I & 0x1000
 & \\
\hline
New
 & S & 0x1000
 & S  &  0x1000
 & S & 0x2000
 &  I & 0x1000
 & P1: Load 0x1000
\cline{2-9}
 & S & 0x1000
 &  I & 0x1000
 &  S & 0x1000
 &  I & 0x1000
 & P2: Load 0x1000 \\
\cline{2-9}
 & S & 0x1000
 & I  & 0x1000
 & S  & 0x2000
 & S  & 0x1000
 & P3: Load 0x1000\\
\hline
\end{tabular}
\end{center}





\begin{parag}{Part I Case 2}
\end{parag}



	\begin{center}
\begin{tabular}{|c|cc|cc|cc|cc|c|}
\hline
 & \multicolumn{8}{c|}{Cache} &  \\
\cline{2-9}
 & \multicolumn{2}{c|}{Cache C0}
 & \multicolumn{2}{c|}{Cache C1}
 & \multicolumn{2}{c|}{Cache C2}
 & \multicolumn{2}{c|}{Cache C3}
 & P\textless n\textgreater : L/S \textless addr\textgreater \\
\cline{2-9}
 & State & Address
 & State & Address
 & State & Address
 & State & Address
 & \\
\hline
Old
 & I & 0x1400
 & I & 0x1200
 & I & 0x1200
 & M & 0x1200
 & \\
\hline
New
 &  & 
 &   & 
 &  & 
 & I & 0x1200
 & \\
\cline{2-9}
 &  & 
 &   &
 &   &
 &  I & 0x1200
 & \\
\cline{2-9}
 &  & 
 &   &
 &   &
 &  I & 0x1200
 & \\
\hline
\end{tabular}
\end{center}
So for this one, we need to go form the Modified state to the invalid state. To do so, we need a BusRdX signal from another processor. To do so, we can do a processor write from an invalid state. 
    
	\begin{center}
\begin{tabular}{|c|cc|cc|cc|cc|c|}
\hline
 & \multicolumn{8}{c|}{Cache} &  \\
\cline{2-9}
 & \multicolumn{2}{c|}{Cache C0}
 & \multicolumn{2}{c|}{Cache C1}
 & \multicolumn{2}{c|}{Cache C2}
 & \multicolumn{2}{c|}{Cache C3}
 & P\textless n\textgreater : L/S \textless addr\textgreater \\
\cline{2-9}
 & State & Address
 & State & Address
 & State & Address
 & State & Address
 & \\
\hline
Old
 & I & 0x1400
 & I & 0x1200
 & I & 0x1200
 & M & 0x1200
 & \\
\hline
New
 & M &  0x1200
 &  I &  0x1200
 & I &  0x1200
 & I & 0x1200
 &  P0: store 0x1200 \\
\cline{2-9}
 & I & 0x1400
 & M  & 0x1200
 & I  & 0x1200
 &  I & 0x1200
 &P1: Store 0x1200 \\
\cline{2-9}
 & I & 0x1400
 &  I & 0x1200
 & M  & 0x1200
 &  I & 0x1200
 & P2: Store 0x1200\\
\hline
\end{tabular}
\end{center}
\begin{parag}{Case 3}
\end{parag}

	\begin{center}
\begin{tabular}{|c|cc|cc|cc|cc|c|}
\hline
 & \multicolumn{8}{c|}{Cache} &  \\
\cline{2-9}
 & \multicolumn{2}{c|}{Cache C0}
 & \multicolumn{2}{c|}{Cache C1}
 & \multicolumn{2}{c|}{Cache C2}
 & \multicolumn{2}{c|}{Cache C3}
 & P\textless n\textgreater : L/S \textless addr\textgreater \\
\cline{2-9}
 & State & Address
 & State & Address
 & State & Address
 & State & Address
 & \\
\hline
Old
 & M & 0x1300
 & I & 0x1200
 & I & 0x1100
 & E & 0x1100
 & \\
\hline
New
 &  & 
 &  & 
 &  & 
 & M & 0x1100
 &  \\
\cline{2-9}
 &  & 
 &   &
 &   &
 &  M & 0x1100
 & \\
\cline{2-9}
 &  & 
 &   &
 &   &
 &  M & 0x1100
 & \\
\hline
\end{tabular}
\end{center}

So the only way to go from the Exclusive clean state to the dirty state is to write something to our value, making it dirty \textrightarrow PrWr.

	\begin{center}
\begin{tabular}{|c|cc|cc|cc|cc|c|}
\hline
 & \multicolumn{8}{c|}{Cache} &  \\
\cline{2-9}
 & \multicolumn{2}{c|}{Cache C0}
 & \multicolumn{2}{c|}{Cache C1}
 & \multicolumn{2}{c|}{Cache C2}
 & \multicolumn{2}{c|}{Cache C3}
 & P\textless n\textgreater : L/S \textless addr\textgreater \\
\cline{2-9}
 & State & Address
 & State & Address
 & State & Address
 & State & Address
 & \\
\hline
Old
 & M & 0x1300
 & I & 0x1200
 & I & 0x1100
 & E & 0x1100
 & \\
\hline
New
 & M &  0x1300
 &  I &  0x1200
 & I &  0x1100
 & M & 0x1100
 & P3: Store 0x1100 \\
\cline{2-9}
 &  & 
 &   &
 &   &
 &  M & 0x1100
 & \\
\cline{2-9}
 &  & 
 &   &
 &   &
 &  M & 0x1100
 & \\
\hline
\end{tabular}
\end{center}

\begin{parag}{Case 4}
    
\end{parag}

	\begin{center}
\begin{tabular}{|c|cc|cc|cc|cc|c|}
\hline
 & \multicolumn{8}{c|}{Cache} &  \\
\cline{2-9}
 & \multicolumn{2}{c|}{Cache C0}
 & \multicolumn{2}{c|}{Cache C1}
 & \multicolumn{2}{c|}{Cache C2}
 & \multicolumn{2}{c|}{Cache C3}
 & P\textless n\textgreater : L/S \textless addr\textgreater \\
\cline{2-9}
 & State & Address
 & State & Address
 & State & Address
 & State & Address
 & \\
\hline
Old
 & I & 0x1200
 & M & 0x1200
 & S & 0x1000
 & S & 0x1000
 & \\
\hline
New
 &  & 
 &  &
 & M & 0x1000
 &  &
 &  \\
\cline{2-9}
 &  & 
 &   &
 &  M & 0x1000
 &  & 
 & \\
\cline{2-9}
 &  & 
 &   &
 &  M & 0x1000
 &   & 
 & \\
\hline
\end{tabular}
\end{center}
As before because we are going from a clean state to a dirty state \textrightarrow we need to write something. The difference here is how would that impact other cache. This time we are putting a BusRdX on the bus which means, if we have it, you may care to know that it is tail. The 0x1200 address doesn't care so they don't change. As for the C3 cache we do care here, our value has been changed now and what we have in our cache is completely useless now... \textrightarrow we go to invalid state:

	\begin{center}
\begin{tabular}{|c|cc|cc|cc|cc|c|}
\hline
 & \multicolumn{8}{c|}{Cache} &  \\
\cline{2-9}
 & \multicolumn{2}{c|}{Cache C0}
 & \multicolumn{2}{c|}{Cache C1}
 & \multicolumn{2}{c|}{Cache C2}
 & \multicolumn{2}{c|}{Cache C3}
 & P\textless n\textgreater : L/S \textless addr\textgreater \\
\cline{2-9}
 & State & Address
 & State & Address
 & State & Address
 & State & Address
 & \\
\hline
Old
 & I & 0x1200
 & M & 0x1200
 & M & 0x1000
 & S & 0x1000
 & \\
\hline
New
 & I & ----
 & M & 0x1200
 & M & 0x1000
 & I & ----
 & P2: Store 1000\\
\cline{2-9}
 &  & 
 &   &
 &  M & 0x1000
 &  & 
 & \\
\cline{2-9}
 &  & 
 &   &
 &  M & 0x1000
 &   & 
 & \\
\hline
\end{tabular}
\end{center}
Donc il y a en a jusqu'a 11, je vais pas tous les copier ici parce que voila... mais ducoup dans le pdf \textit{5.b Examples of Cache Coherence}.

