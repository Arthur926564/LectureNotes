% !TeX program = lualate
% Using VimTeX, you need to reload the plugin (\lx) after having saved the document in order to use LuaLaTeX (thanks to the line above)

\documentclass[a4paper]{article}

% Expanded on !p snip.rv = get_formatted_date() at !p snip.rv = get_formatted_time().

\usepackage{style}

\title{corrigé exa 2024}
\author{Arthur Herbette }
\date{Mardi 28 octobre 2025}

\begin{document}
\maketitle
\begin{parag}{SCQ 1}
    Je possède une using consommant du gaz (coût) sur quelle coût dois-je prendre mes décisions:
    \begin{itemize}
        \item Carburant \textrightarrow \important{oui} coût continue
        \item Compensation carbone et Opération et mangamenet \textrightarrow oui coût continu
        \item Investissement et Démantèlement \textrightarrow \important{non} investissement non continu
    \end{itemize}
\end{parag}
\begin{parag}{SCQ 2}
    On parle d'un type de bien économique qu'on peut dificilement empêcher les gens d'utiliser (non-excluable) \important{et} dont l'usage par une persone \important{ne réduit pas} (ou peu) la possibilité des autres d'en profiter
    \begin{itemize}
          \item Bien collectif (Défense national) Oui et Oui
          \item Ressources communes (poissons) Oui et non (si je prends tout les poissons il y en a plus pour toi)
          \item Bien de club (match de foot) Non et ?
          \item Bien privé que pour moi donc non
    \end{itemize}
\end{parag}
\begin{parag}{SCQ 3}
    \begin{itemize}
          \item Valeur nominal \textrightarrow ne prends pas en compte l'inflation Montant affiché 'raw' en francs d'aujourd'hui
          \item Valeur réel \textrightarrow prends en compte l'inflation, Interprète avec le pouvoir d'achat, valeur corrigé en fonction de l'évolution des prix
    \end{itemize}
    je crois que c'est l'inflation au lieu de ination
    \begin{itemize}
          \item Inflation seul ne prends en compte que la valeur réelle donc oui
          \item Non ducoup (taux d'acutalisation)
          \item Valeur temps \textrightarrow décris l'évolution de l'argent au court du temps donc non
          \item non plus
    \end{itemize}
\end{parag}

\begin{parag}{SCQ 4}
    Pas vu
\end{parag}
\begin{parag}{SCQ 5}
    Sunk cost fallcy, ce qui a été vu en cours et aussi demandé dans la question 1
\end{parag}


\begin{parag}{scq 6}
      Le donut dit qu'il faut toruver un just milieux entre notre societé (fondations sociales) et les limites planétaire, le but est quand même d'avoir un meilleur niveau de vie mais en respectant les limites planétaires.
\end{parag}

\begin{parag}{scq 7}
    Le taux d'actualisation veut dire que chaque anne ce qu'on a acheter perd 5 pourccent de sa valeur\\
    Pour calculer, on prends d'abord l'argent qu'on a récuperer et on le 'traduit' en la valeur lors de l'achat pour faire sa on fait:
    \begin{align*} \frac{A}{\left(1 + t\right)^{n}} \end{align*}
    ou t est le taux d'actualisation et $n$ le nombre d'année.\\
    Donc dans le premier cas on a que la valeur de notre investissement au bouts de 20 ans est:
    \begin{align*} \frac{1400}{\left(1.05\right)^{20}} = 528 \end{align*}
    Ce qui est donc rentable ($528 > 500$).
\end{parag}
\begin{parag}{SCQ8}
    On a la courbe noire qui represente la consommation du pays durant l'année, on voir que les mois de janvier, fevrier, mars, novembre, décembre la consomation du pazs est plus haute que la production de ce derniers. Dès lors le pays a forcémment du importé de l'électricité en hiver.\\
    Pour le reste on a:
    \begin{itemize}
          \item "On a importé de l'électricité en hiver": faux, On a produit plus que la consommation donc non
          \item 'La consommation est la plut fort en juillet': faux, c'est littéralement le mois avec la consomation la plus basse
          \item 'La production de nucléaire était la plus basse en février' non la production en jui est plus basse.
    \end{itemize}
   
\end{parag}
\begin{parag}{scq 9}
\end{parag}
\begin{parag}{scq 10}
    Donc ici on a que la courbe rouge est la \important{demande} et la courbe bleu est \important{l'offre}. Lorsqu'il y a une diminution de la demande on a alors que la quantité voulu baise, dès lors notre courbe rouge dois se décaler vers la gauche.\\
    Les graphe a et b nous parle du changement de l'offre.
\end{parag}

\end{document}
