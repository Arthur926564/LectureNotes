\lecture{2}{2024-09-12}{Vecteur}{}
\begin{parag}{Espaces vectoriels}
    \begin{definition}
        Un \important{espace vectoriel} est un ensemble $V$ non vide dont les éléments sont appelés \important{vecteurs}. Il est muni de deux opérations
        \begin{itemize}
            \item \textbf{addition} $+ : V \times V \to V$ qui associe deux vecteurs $(u, v)$ leur somme $u + v$
            \item \textbf{action} \R$ \times V \to V$ qui associe à un nombre $\alpha$ et un vecteur $u$ leur produit $\alpha u$
        \end{itemize}
    \end{definition}
    \begin{subparag}{Espaces vectoriels : axiomes}
        \begin{enumerate}
            \item commutativité de $+ : u + v = v + u$
            \item associativité de $+ : (u+v) + w = u + (v + w)$
            \item vecteur nul: il existe un élément $0$ de $V$ tel que $u  +0 = u$
            \item opposé: il existe un vecteur $-u$ de $V$ tel que $u + (-u) = 0$
            \item distributivité 1: $\alpha(u + v) = \alpha u + \alpha v$
            \item distributivité 2: $(\alpha + \beta) u = \alpha u + \beta u$
            \item "\textit{compatibilité}": $(\alpha\beta)u = \alpha(\beta u)$
            \item unité: $1 \cdot u = u$
        \end{enumerate}
    \end{subparag}
    \begin{framedremark}
        Donc un espace vectoriel peut être un peu n'importe quoi tant que ces règles sont respectés, cela pourrait être des fonctions, des polynômes, des suites etc...
    \end{framedremark}
\end{parag}